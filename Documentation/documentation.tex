\documentclass[]{report}
\usepackage{polski}
\usepackage[utf8]{inputenc}


% Title Page
\title{ Rozpoznawanie i klasyfikacja pisanych cyfr przy użyciu modeli matematycznych  }
\author{Anna Zawadzka \\ Piotr Waszkiewicz}


\begin{document}
\maketitle

\chapter{Projekt}


\section{Opis}

Celem zadania jest porównanie jakości klasyfikacji dla różnych modeli matematycznych, oraz próba minimalizacji ich błędów poprzez ekstrakcję dodatkowych cech obiektów z bazy danych. Podczas realizacji projektu wykorzystane zostaną jedne z najpopularniejszych obecnie klasyfikatorów: maszyny wektorów podpierających (SVM), Lasy Losowe, kNN, model regresji wielomianowej oraz sieci neuronowe. Zbiory danych treningowych oraz testowych zostaną zaczerpnięte z publicznej bazy danych MNIST\cite{mnist_database}. \\


\section{Cel badań}

Celem badania jest wskazanie najskuteczniejszego klasyfikatora pod względem czasu uczenia, wydajności i jakości udzielanych odpowiedzi. Oprócz tego badania mają na celu rozszerzenie istniejącego wektora cech o nowe, unikalne wartości które polepszą jakość klasyfikacji. Przykładem takich cech może być liczba przecięć w napisanym symbolu, liczba zakończeń lub procent powierzchni zajmowanej przez narysowany symbol. W trakcie obliczeń podjęta zostanie próba odrzucenia tych cech które przeszkadzają lub pogarszają działanie modeli. Przeprowadzone badania obejmą również wybór optymalnych parametrów dla poszczególnych klasyfikatorów metodą GridSearch\cite{gridsearch}. \\


\section{Zbiory danych}

Zbiory danych treningowych oraz testowych pochodzą z publicznej bazy danych MNIST\cite{mnist_database}. Każdy element ze zbioru treningowego jest opisany 785 wartościami. Pierwsza liczba określa zakodowaną cyfrę (wartość z przedziału [0, 9]), kolejne 784 wartości są z przedziału [0, 255] i opisują kolory pikseli zeskanowanej cyfry w skali szarości dla obrazka o wymiarach 28x28 pikseli. Zbiór testowy w przeciwieństwie do treningowego nie zawiera informacji o reprezentowanej klasie. Zbiór treningowy i testowy zawierają odpowiednio 60,000 i 10,000 elementów. \\


\section{Sposób weryfikacji rozwiązań}

Podczas ewaluacji otrzymywanych rozwiązań minimalizowana będzie funkcja błędu opisana wzorem \[ f(M, d) = e(M, d) + t(M, d) \] gdzie \textit{M} oznacza model, \textit{d} zbiór testowy, \textit{e()} współczynnik \textit{Error rate}, czyli miarę określającą stosunek źle zaklasyfikowanych elementów do wszystkich obiektów w zbiorze, oraz \textit{t()} funkcję czasu liczoną jako liczbę sekund potrzebną na realizację obliczeń.

\begin{thebibliography}{9}
	\bibitem{mnist_database} http://yann.lecun.com/exdb/mnist/index.html
	\bibitem{gridsearch} http://scikit-learn.org/stable/modules/grid\_search.html
\end{thebibliography}

\end{document}          
